\documentclass[12pt,a4paper]{article}
\usepackage{ctex}          % 中文支持
\usepackage{geometry}      % 页面布局
\usepackage{amsmath,amssymb} % 数学公式
\usepackage{fancyhdr}      % 页眉页脚
\usepackage{multirow}      % 表格合并单元格
\usepackage{graphicx}      % 插入图片
\usepackage{enumitem}      % 自定义列表
\usepackage{titlesec}      % 自定义标题格式

% 页面设置
\geometry{left=2.5cm, right=2.5cm, top=2.5cm, bottom=2.5cm}


% 标题样式
\titleformat{\section}{\large\bfseries}{\chinese{section}、}{0.5em}{}
\titleformat{\subsection}{\normalsize\bfseries}{(\chinese{subsection})}{0.5em}{}

\begin{document}

\begin{center}
	\includegraphics[width=4cm]{./cover/title.pdf}
\end{center}
\begin{center}
    \zihao{-3}计算机学院 2024-2025 学年第1学期期末考试\\[5pt]
    \zihao{-3}《信息安全数学基础》课程试卷(B)
\end{center}

\begin{center}
\begin{flushleft}
    专业:\underline{\hspace{2cm}}
    年级:\underline{\hspace{1.5cm}}
    班级:\underline{\hspace{1.5cm}}
    姓名:\underline{\hspace{1.5cm}}
    学号:\underline{\hspace{2.5cm}}
\end{flushleft}
\end{center}
\begin{center}
\begin{tabular}{|c|c|c|c|c|c|c|}
    \hline
    题号 & 一 & 二 & 三 & 四 & 五 & 总分 \\
    \hline
    得分 & & & & & & \\
    \hline
\end{tabular}
\end{center}

\vspace{10pt}

% 一、计算题
\section{计算题(30分)}

1. 求2023与503的最大公因子。(3分)

2. 通过计算判断 $10040 + 4873$ 是否可以被37整除。(5分)

3. 已知模数 $p=131$,请问乘法群 $\mathbb{Z}_p^*$ 有多少个生成元,请给出计算过程。(5分)

4. 给定 $p=17$, $q=11$, $n=p \times q$。已知 $m=5$,请使用中国剩余定理求 $m^{23} \mod n$。请详细给出计算过程。(7分)

5. 给定 $p=127$, 且 $a=3$ 是模 $p$ 的原根,请计算给出另一个原根,并说明理由。(5分)

6. 求解同余方程:$23x \equiv 24 \pmod{37}$,给出过程。(5分)

% 二、证明题
\section{证明题(40分)}

1. 证明:任意阿贝尔群 $G$ 的有限阶元素形成群 $G$ 的子群。(10分)

2. 给定任意群 $G$,$H$ 是群 $G$ 的正规子群。请证明:如果群 $G$ 是循环群,则商群 $G/H$ 也是循环群。(10分)

3. 设 $n=pq$,$p$ 和 $q$ 是两个不相同的素数,请证明 $\mathbb{Z}_n^* \cong \mathbb{Z}_p^* \times \mathbb{Z}_q^*$。(10分)

4. 证明:$2\mathbb{Z} + 3\mathbb{Z} = \mathbb{Z}$。(10分)

% 三、问答题
\section{问答题(10分)}

1. 设 $p$ 是奇素数,请解释为什么 $\mathbb{Z}_p^*$ 的所有生成元都是模 $p$ 的二次非剩余。(5分)

2. 设 $G$ 是阶为 $pq$ 的阿贝尔群,其中 $p$ 和 $q$ 是两个不同的素数。已知群元 $g_1$ 的阶为 $p$,群元 $g_2$ 的阶为 $q$,请问群元 $g_1g_2$ 的阶是多少?请说明理由。(5分)

% 四、算法描述题
\section{算法描述题(10分)}

请描述一种求乘法逆元的迭代算法过程。即,给定正整数 $a$ 和 $n$,且 $a$ 与 $n$ 互素,求正整数 $b$ 使得 $ab \equiv 1 \pmod{n}$。并说明该算法的时间复杂性。

% 五、应用题
\section{应用题(10分)}

RSA 是一种著名的公钥加密算法,简单来说,它的功能是将 $\mathbb{Z}_n^*$ 中的群元映射为 $\mathbb{Z}_n^*$ 上另一个群元,且必须保证这种映射是单射。给定一个 1024 比特的大整数 $n$ 和一个整数 $e=65537$,这种加密算法定义的映射为:$\phi: \mathbb{Z}_n^* \to \mathbb{Z}_n^*$,对任意 $m \in \mathbb{Z}_n^*$,$\phi(m) = m^e$。请问在什么条件下 $\phi$ 会是一种单射?请论证并说明理由。

\vspace{20pt}
\hrule
\vspace{10pt}

% 答题卡部分
\newpage
\begin{center}
	\includegraphics[width=4cm]{./cover/title.pdf}
\end{center}

\begin{center}
    \zihao{-3}计算机学院 2024-2025 学年第1学期期末考试\\[5pt]
    \zihao{-3}《信息安全数学基础》课程试卷(B)答题纸
\end{center}
% 基本信息表格
\begin{center}
\begin{flushleft}
    专业:\underline{\hspace{2cm}}
    年级:\underline{\hspace{1.5cm}}
    班级:\underline{\hspace{1.5cm}}
    姓名:\underline{\hspace{1.5cm}}
    学号:\underline{\hspace{2.5cm}}
\end{flushleft}
\end{center}

\vspace{5pt}

% 题号与得分表格
\begin{center}
\begin{tabular}{|c|c|c|c|c|c|c|}
    \hline
    题号 & 一 & 二 & 三 & 四 & 五 & 总分 \\
    \hline
    得分 & \makebox[1.5cm]{\hfill} & \makebox[1.5cm]{\hfill} & \makebox[1.5cm]{\hfill} & \makebox[1.5cm]{\hfill} & \makebox[1.5cm]{\hfill} & \makebox[1.5cm]{\hfill} \\
    \hline
\end{tabular}
\end{center}
% 答题区
\section*{一、计算题(30分)}
\section*{二、证明题(40分)}
\section*{三、问答题(10分)}
\section*{四、算法描述题(10分)}
\section*{五、应用题(10分)}

\end{document}