\section{Chap1}
\label{20250215}

\subsection{公钥加密方案}
\begin{description}
\item[RSA加密] 基于因子分解/RSA假设,分解大数是个难题
\item[ElGamal加密] 基于离散对数/Diffie-Hellman假设,即在$\ZZ_p$中找到离散对数很困难
\item[基于格的加密] 前两种方案不具备量子安全性。量子计算将破解这些方案。基于格的加密方案是后量子安全(post-quantum)的。它们与在格中找到“短”向量的难题相关。
\end{description}

数论背景知识
\begin{definition}
    我们用$a\mid b$表示$a$ 整除 $b$,即存在整数$c$使得$b = a\cdot c$
\end{definition}
\begin{definition}
    $\gcd(a, b)$表示$a, b$的最大公约数。如果$\gcd(a, b) = 1$,则$a, b$互质。
\end{definition}
\begin{ques*}
    我们如何计算$\gcd$?它的时间复杂度是多少?
\end{ques*}
\begin{example*}
    我们使用欧几里得算法。以计算$\gcd(12, 17)$为例,
    \begin{align*}
        17\mod{12} & = 5 \\
        12 \mod{5} & = 2 \\
        5\mod{2}   & = 1 \\
        2\mod{1}   & = 0
    \end{align*}
    或计算 $\gcd(12, 18)$
    \begin{align*}
        18\mod{12} & = 6 \\
        12\mod{6}  & = 0
    \end{align*}
\end{example*}
如果我们有两个长度为$n$比特的比特串,欧几里得算法的运行时间是多少?

粗略的说,我们可以看到,在每一步中,$a, b$的长度大约减少1比特。因此,计算$\gcd$的时间复杂度大致为$O(n)$。


\begin{definition}[模运算]
$a\mod{N}$是$a$除以$N$的余数
	当$a$和$b$对模$N$同余时,记作$a\equiv b\pmod{N}$。也就是说,$a\mod N = b\mod N$。
\end{definition}

朴素的方法是重复相乘。但这需要$b$步骤($2^n$次乘法)。

我们可以“重复平方”。例如,我们可以通过得到$a^2$,再平方得到$a^4$,再平方得到$a^8$,从而更快地计算$a^8$。我们可以利用比特串$b$来确定如何计算这个值。

\begin{example*}
    如果$b = 100101_2$,我们可以计算$a\cdot a^4\cdot a^{32}\mod N$,这可以通过递归计算!
\end{example*}

这个算法的时间复杂度对于$n$比特的$a, b, N$是$O(n)$\footnote{不完全是$O(n)$,我们应该加上乘法的复杂度。然而,由于我们可以在每一步中取对数,所以这个复杂度应该受到$N$的限制。}。

\begin{theorem}[Bezout's Theorem, \emph{roughly}]
    如果$\gcd(a,N) = 1$,则$\exists b$使得
    \[a\cdot b \equiv 1\pmod{N}.\]
	这意味着$a$在模$N$下是可逆的。$b$是$a$的逆元,记作$a^{-1}$。
\end{theorem}

\begin{ques*}
    如何计算$b$?
\end{ques*}
我们可以使用扩展(extended)欧几里得算法(EGCD)!
\begin{example*}
    我们使用线性方程组表示$a$和$N$,使它们的和等于$1$,使用之前的欧几里得算法。以前面的例子$\gcd(12, 17)$为例,
    \begin{align*}
        17\mod{12} & = 5 \\
        12 \mod{5} & = 2 \\
        5\mod{2}   & = 1 \\
        2\mod{1}   & = 0
    \end{align*}
    我们将其写成
    \begin{align*}
        5 & = 17 - 12\cdot 1                         \\
        2 & = 12 - 5\cdot 2 = 12\cdot x + 17\cdot y  \\
        1 & = 5 - 2\cdot 2 = 12\cdot x' + 17\cdot y'
    \end{align*}
    在第2行中,我们将$5$的线性组合代入第2行中的$2$,在第1行中,我们将$2$的线性组合代入第1行中,从而产生了另一个$12, 17$的线性组合

    如果$\gcd(a, N) = 1$,我们使用egcd算法将$1 = a\cdot x + N\cdot y$写成标准形式,然后得到$1\equiv a\cdot x\pmod{N}$。
\end{example*}

\begin{definition}[模$N$的单位群]
    我们定义集合
    \[\ZZ_N^\times:= \left\{ a\mid a\in [1, N-1], \gcd(a, N) = 1 \right\}\]
    这是模$N$的单位群(它们都是单位元,因为根据上述内容它们都有逆元)
\end{definition}
\begin{definition}[欧拉函数]
    欧拉函数$\phi(N)$计算该集合中的元素数量。也就是说,$\phi(N) = |\ZZ_N^\times|$
\end{definition}
\begin{theorem}[欧拉定理]
	对于所有满足$\gcd(a, N) = 1$的$a, N$,我们有\[a^{\phi(N)} \equiv 1\pmod{N}.\]
\end{theorem}

有了这些定理,再来看看RSA。

\subsection{RSA}
RSA是一个公钥加密方案。它的安全性基于因子分解假设。
\begin{definition}[因子分解假设]
	给定两个$n$比特的质数$p, q$,我们计算$N = p\cdot q$。在给定$N$的情况下,(经典地)寻找$p$和$q$是计算上困难的。
\end{definition}
\begin{definition}[RSA Assumption]
    假设给定两个$n$比特的质数,我们计算$N = p\cdot q$,其中$\phi(N) = (p-1)(q-1)$。我们选择一个满足$\gcd(e, \phi(N)) = 1$的$e$,并计算$d \equiv e^{-1}\pmod{\phi(N)}$。

    给定$N$和一个随机选择的$y\in \ZZ_{N}^\times$,在计算上很难找到一个$x$使得$x^e\equiv y\pmod{N}$。

	但是,如果给定$p$和$q$,则很容易找到$d$。我们知道$\phi(N) = (p-1)(q-1)$,所以我们可以通过运行egcd算法来从$e$计算$d$。然后,我们取$(x^e)^d\equiv x^{ed}\equiv x$,从而可以再次提取$x$。

\end{definition}

加密就是将明文进行$d$次幂运算,解密则是再次进行$e$次幂运算。

剩下的问题:
\begin{itemize}
\item 如何生成质数$p$和$q$?
\item 如何选择满足$\gcd(e, \phi(N)) = 1$的$e$?
\item 有哪些安全问题?
\end{itemize}